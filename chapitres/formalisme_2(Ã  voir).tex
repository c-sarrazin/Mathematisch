\titre{Les Bases}{Vocabulaire, notations et premiers raisonnement}
\subsection{Bases de raisonnement et raisonnement basique}
\subsubsection{L'alphabet grec}

\begin{table}[]
\begin{tabular}{|lll|}
\hline
\begin{tabular}[c]{@{}l@{}}$\alpha$ A alpha\\ $\beta$ $Beta$ bêta\\ $\gamma$ $Gamma$ gamma\\ $\delta$ $\Delta$ delta\\ $\varepsilon$ E epsilon\\ $\zeta$ Z dzeta\\ $\eta$ H êta \\ $\theta$ $\Theta$ thêta\end{tabular}
 & \begin{tabular}[c]{@{}l@{}}$\iota$ I iota\\ $\kappa$ K kappa\\ $\lambda$ $\Lambda$ lambda\\ $\mu$ M mu\\ $\nu$ N nu\\ $\xi$ $\Xi$ xi\\ o O omicron\\ $\pi$ $\Pi$ pi\end{tabular} 
 & \begin{tabular}[c]{@{}l@{}}$\rho$ P rhô\\ $\sigma$ $\Sigma$ sigma\\ $\tau$ T tau\\ $\upsilon$ Y upsilon\\ $\phi$ $\Phi$ phi\\ $\chi$ X khi\\ $\psi$ $\Psi$ psi\\ $\omega$ $\Omega$ oméga\end{tabular} \\
\hline \end{tabular}
\end{table}

\subsubsection{Raisonnement}
\begin{defini}[Proposition]
    Une proposition est un énoncé mathématique qui peut prendre la valeur de vérité : vrai ou faux.
\end{defini}
\begin{ex}
    \textit{0 est un nombre pair}; est une proposition dont la valeur logique est vrai.

    \textit{0 > 1}; est une proposition dont la valeur est faux.
\end{ex}

\begin{table}
    \centering
    \arrayrulecolor[rgb]{0.502,0.502,0.502}
    \begin{tabular}{!{\color{black}\vrule}c!{\color{black}\vrule}c!{\color{black}\vrule}c|c|c|c|c!{\color{black}\vrule}} 
    \arrayrulecolor{black}\hline
    $P$ & $Q$ & $P \land Q $ & $P \vee Q$ & $P \implies Q$ & $P \iff Q$ & $\lnot P$  \\ 
    \hline
    V   & V   & V            & V          & V              & V          & F          \\ 
    \arrayrulecolor[rgb]{0.502,0.502,0.502}\hline
    V   & F   & F            & V          & F              & F          & F          \\ 
    \hline
    F   & V   & F            & V          & V              & F          & V          \\ 
    \hline
    F   & F   & F            & F          & V              & V          & V          \\
    \arrayrulecolor{black}\hline
    \end{tabular}
    \end{table}
