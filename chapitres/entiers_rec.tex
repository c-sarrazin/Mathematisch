\titre{Entiers, principe de récurrence et suites}{Encore de la logique et de la théorie des ensembles}  \label{card}

Une note enfin pour conclure ces trois chapitres "fondationnels" : on a choisi pour base de l'édifice mathématique la théorie des ensembles. Il s'agit d'un choix éminemment discutable, qui a surtout à voir avec la tradition et l'affinité des auteurs pour ce cadre. D'autres théories, comme la théorie des catégories ou le $\lambda$-calcul, se prêtent tout aussi bien à la tâche, voire mieux pour certains usages (notamment pour l'algèbre ou l'informatique théorique). Il n'y a aucune raison d'affirmer que le nombre deux \emph{est} l'ensemble $\{\varnothing, \{\varnothing\}\}$ : ces ensembles et les opérations qu'on a définies dessus sont simplement une structure (on parle de modèle) sur laquelle les énoncés intuitifs de l'arithmétique sont vrais. Cœur cœur pluralisme tolérance etc.