\begin{multicols*}{2}
    \titre{Construction de $\mathbb{R}$}{Suites de Cauchy, coupures de Dedekind, théorèmes fondamentaux pour l'analyse}
    \subsection{Motivation et suites de Cauchy}
\subsubsection{Pourquoi $\mathbb{R}$ ?}
Qu'est ce que l'ensemble des réels ? 
Intuitivement, c'est l'ensemble des nombres rationnels dont on a "rempli les trous". Mais que sont donc ces trous ? Par exemple, une solution de $x^2=2$ :

\medskip
\begin{tcolorbox}[coltitle=black, colframe=blue!50!green!20!white, colback=white, adjusted title=Une preuve de l'irrationnalité de $\sqrt{2}$]
    On suppose qu'il existe deux entiers $p,q$ premiers entre eux tels que $\left(\frac{p}{q}\right)^{2}=2$. Alors $p^2=2q^2$, donc $p^2$ est pair. Mais tout entier ayant la même parité que son carré, $p$ est également pair. Avec $p=2k$, il vient $4k^2=2q^2$, d'où $2k^2=q^2$, et rebelote : $q$ est pair.

    On avait supposé la fraction irréductible, et pourtant $\mathrm{PGCD}(p,q) \geq 2...$. C'est impossible, donc $\sqrt{2}$ est irrationnel.
\end{tcolorbox}
\medskip

\noindent Comment faire sens alors d'une telle solution ?

Peut être d'une façon approchée : par exemple, en construisant une suite de rationnels dont le carré converge vers 2. 

\begin{exo}[Méthode de Héron pour l'approximation de $\sqrt{2}$]
    Help
\end{exo}

\end{multicols*}