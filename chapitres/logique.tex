\titre{Les Bases}{Vocabulaire, notations et premiers raisonnement}
\subsection{Bases de raisonnement et raisonnement basique}
\subsubsection{L'alphabet grec}

\begin{center}
\begin{tabular}{|lll|}
\hline
\begin{tabular}[c]{@{}l@{}}$\alpha$ A alpha\\ $\beta$ B bêta\\ $\gamma$ $\Gamma$ gamma\\ $\delta$ $\Delta$ delta\\ $\varepsilon$ E epsilon\\ $\zeta$ Z dzeta\\ $\eta$ H êta \\ $\theta$ $\Theta$ thêta\end{tabular}
 & \begin{tabular}[c]{@{}l@{}}$\iota$ I iota\\ $\kappa$ K kappa\\ $\lambda$ $\Lambda$ lambda\\ $\mu$ M mu\\ $\nu$ N nu\\ $\xi$ $\Xi$ xi\\ o O omicron\\ $\pi$ $\Pi$ pi\end{tabular} 
 & \begin{tabular}[c]{@{}l@{}}$\rho$ P rhô\\ $\sigma$ $\Sigma$ sigma\\ $\tau$ T tau\\ $\upsilon$ Y upsilon\\ $\phi$ $\Phi$ phi\\ $\chi$ X khi\\ $\psi$ $\Psi$ psi\\ $\omega$ $\Omega$ oméga\end{tabular} \\
\hline \end{tabular}
\end{center}

\subsubsection{Raisonnement}
\label{lang}
\begin{defini}[Proposition]
    Une proposition est un énoncé mathématique qui peut prendre la valeur de vérité : vrai ou faux.
\end{defini}
\begin{ex}
    \textit{0 est un nombre pair}; est une proposition dont la valeur logique est vrai.

    \textit{0 > 1}; est une proposition dont la valeur est faux.
\end{ex}
Soit $P,Q,R$ des variables propositionnelles, elles représentent une proposition.
\begin{defini}[Connecteurs(liens) logiques usuels ]
    Les connecteurs logiques permettent de crée des nouvelles propositionnelles

    $\land$ le "et", $P\land Q$ prend la valeur vraie si $P$ est vraie et $Q$ est vraie.

    $\vee$ le "ou"(inclusif), $P\vee Q$ prend la valeur vraie si au moins l'un des $P$ ou $Q$ prend la valeur vraie

    $\implies$ le signe d'implication, $P\implies Q$ prend la valeur vraie si lorsque $P$ est vraie alors $Q$ est vraie. Attention si $P$ est fausse alors $P\implies Q$ est vraie quelque soit la valeur de $Q$,comme le dit l'adage "Avec des si on mettrait Paris en bouteille"

    $\iff$ le signe d'équivalence, $P\iff Q$ prend la valeur vraie si $P$ et $Q$ on la toujours la même valeur.

    $\lnot$ le "non", $\lnot P$ prend la valeur vraie si $P$ est fausse.
    
\end{defini}
Table de verité des liens logiques usuels 
\begin{center}
    \arrayrulecolor[rgb]{0.502,0.502,0.502}
    \begin{tabular}{!{\color{black}\vrule}c!{\color{black}\vrule}c!{\color{black}\vrule}c|c|c|c|c!{\color{black}\vrule}} 
    \arrayrulecolor{black}\hline
    $P$ & $Q$ & $P \land Q $ & $P \vee Q$ & $P \implies Q$ & $P \iff Q$ & $\lnot P$  \\ 
    \hline
    V   & V   & V            & V          & V              & V          & F          \\ 
    \arrayrulecolor[rgb]{0.502,0.502,0.502}\hline
    V   & F   & F            & V          & F              & F          & F          \\ 
    \hline
    F   & V   & F            & V          & V              & F          & V          \\ 
    \hline
    F   & F   & F            & F          & V              & V          & V          \\
    \arrayrulecolor{black}\hline
    \end{tabular}
\end{center}
\begin{exo}
    Montrer à l'aide de la table de vérité que $(P\iff Q)\iff((P\implies Q)\land(Q\implies P))$
    Ceci justifie que pour montrer une équivalence, on montre en général deux implications.
\end{exo}
\begin{defini}[Condition nécessaire, suffisante]
    On dit que $P$ est une condition suffisante pour avoir $Q$ si $P\implies Q$

    On dit que $Q$ est une condition nécessaire pour avoir $P$ si $P\implies Q$

    On dit que $Q$ est une condition néccessaire et suffisante pour avoir $P$ si $P\iff Q$
\end{defini}