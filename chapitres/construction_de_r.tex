\titre{Construction de $\mathbb{R}$ et de $\mathbb{C}$}{Suites de Cauchy, coupures de Dedekind, théorèmes fondamentaux pour l'analyse}
\subsection{Motivation et suites de Cauchy}
\subsubsection{Pourquoi $\mathbb{R}$ ?}
Qu'est ce que l'ensemble des réels ? 
Intuitivement, c'est l'ensemble des nombres rationnels dont on a "rempli les trous". Mais que sont donc ces trous ? Par exemple, une solution de $x^2=2$ :

\medskip
\begin{tcolorbox}[coltitle=black, colframe=blue!50!green!20!white, colback=white, adjusted title=Une preuve de l'irrationnalité de $\sqrt{2}$]
    On suppose qu'il existe deux entiers $p,q$ premiers entre eux tels que $\left(\frac{p}{q}\right)^{2}=2$. Alors $p^2=2q^2$, donc $p^2$ est pair. Mais tout entier ayant la même parité que son carré, $p$ est également pair. Avec $p=2k$, il vient $4k^2=2q^2$, d'où $2k^2=q^2$, et rebelote : $q$ est pair.

    On avait supposé la fraction irréductible, et pourtant $\mathrm{PGCD}(p,q) \geq 2...$. C'est impossible, donc $\sqrt{2}$ est irrationnel.
\end{tcolorbox}
\medskip

Comment faire sens alors d'une telle solution ?

Peut être d'une façon approchée : par exemple, en construisant une suite de rationnels dont le carré converge vers 2. 

\begin{exo}[Méthode de Héron pour l'approximation de $\sqrt{2}$]
    On définit par récurrence la suite rationnelle suivante :
$$\begin{cases}
u_0=2\\
u_{n+1}= \frac{1}{2}(u_n+\frac{2}{u_n})\\
\end{cases}$$
\begin{enumerate}[font=\color{blue!50!green}]
    \item Montrer que $u_n^2>2$.
    \item Montrer (sans utiliser le théorème de la limite monotone, puisqu'il n'est pas valable pour des suites rationelles) que la suite définie par $v_0=2$ et $v_{n+1}=(\frac{v_n}{2})^2$ tend vers 0.
    \item Montrer que $u_n^2 \to 2$, en procédant par majoration de $u_n^2-2$ par $v_n$.
\end{enumerate}
\end{exo}

Les termes de cette suite sont successivement, en valeur approchée, [...]. Ils semblent être de plus en plus proches les uns des autres. On peut formaliser cette notion.

\begin{defini}
On dit qu'une suite $(u_n)$ est de Cauchy quand :
$$\forall \varepsilon > 0, \exists N \in \mathbb{N}, (m \geq N \text{ et } n \geq N) \implies |u_n-u_m| < \varepsilon$$
Intuitivement, cela veut dire que les termes sont de plus en plus proches deux à deux.

On notera $\mathcal{C}_{\mathbb{Q}}$ l'ensemble des suites de Cauchy rationnelles.
\end{defini}

\begin{prop}
Toute suite convergente est de Cauchy.
\tcblower
On rappelle que $u_n \to l$ quand :
$$\forall \varepsilon > 0, \exists N \in \mathbb{N}, n \geq N \implies |u_n-l| < \varepsilon$$
Soit $u_n \to l$ et $\varepsilon > 0$.
$$\exists N \in \mathbb{N}, n \geq N \implies l-\varepsilon/2 < u_n < l+\varepsilon/2$$
Alors si $n \geq N$et $m \geq  N$ :
$$l-\varepsilon/2 < u_n < l+\varepsilon/2$$
$$l-\varepsilon/2 < u_m < l+\varepsilon/2$$
D'où :
$$-\varepsilon < u_n-u_m < \varepsilon$$
Autrement dit, $|u_n-u_m|<\varepsilon$ et donc $(u_n)$ est de Cauchy.
\end{prop}    

On va immédiatemment montrer que la réciproque est fausse dans $\mathbb{Q}$.

\begin{exo}
La suite $(u_n)$ est celle définie précédemment.
\begin{enumerate}[font=\color{blue!50!green}]
\item En se souvenant que $u_n^2>2$, montrer que $(u_n)$ décroit.
\item En se souvenant que $u_n^2 \to 2$, déduire que $\displaystyle\forall p,\lim_{n\to+\infty} |u_{n+p}-u_n| = 0$.
\item En conclure que $(u_n)$ est de Cauchy.
\end{enumerate}
\end{exo}

Vue depuis le monde rationnel, cette suite n'est pourtant pas convergente, puisque $\sqrt{2}$ est irrationelle. Ceci nous fournit un contre-exemple à la réciproque de la propriété 1. 
Pourtant, les termes semblent bien se rapprocher "de quelque chose" : ce quelque chose, c'est le nombre réel $\sqrt{ 2 }$, qu'il reste encore à définir.

\subsubsection{Quelques propriétés des suites de Cauchy}

Avant d'attaquer la construction, on montre ici quelques propriétés qu'il sera utile d'avoir en tête :

\begin{prop}
Toute suite de Cauchy est bornée.
\tcblower
Soit $\varepsilon>0$. Il existe un rang $N$ tel que si $n,m \geq N$ alors $|u_n-u_m|<\epsilon$. En particulier, $|u_N-u_n|<\varepsilon$, c'est à dire $u_n \in [u_N - \varepsilon, u_N + \varepsilon]$. Mais alors $\forall n \in \mathbb{N}, u_n \leq \max (\{ u_k | k < N\} \cup \{u_N+\varepsilon\})$, et de même $u_n \geq \min (\{ u_k | k < N\} \cup \{u_N-\varepsilon\})$. Finalement, $(u_n)$ est majorée et minorée, donc bornée.
\end{prop}

\begin{prop}
    Toute suite de Cauchy ne convergeant pas vers 0 est non nulle à partir d'un certain rang.
    \tcblower
    Supposons l'inverse : pour tout $N \in \mathbb{N}$ aussi grand soit-il, il existe un $n\geq N$ tel que $u_n=0$. Soit $\varepsilon >0$. Il existe $N \in \mathbb{N}$ tel que si $m>N$ et $p>N$, $|u_m-u_p|<\varepsilon$. Soit $n>N$ avec $u_n=0$ : alors $\forall m \geq N$, $|u_m-u_n|<\varepsilon$ soit $|u_m|<\varepsilon$. D'où $u_n \to 0$. 
\end{prop}

\begin{theoreme}[Analogue au théorème de la limite monotone]
Toute suite rationnelle monotone bornée est de Cauchy.
\tcblower
On fait une démonstration par dichotomie dans le cas croissante et majorée.

Soit $(u_n) \in \mathbb{Q}^{\mathbb{N}}$ croissante et majorée par un rationnel $M$. Pour tout $n$, $u_0\leq u_n \leq M$. On pose $a_0=u_0$ et $b_0=M$.

On va construire par récurrence deux suites :
\begin{itemize}
    \item Si $[a_n, \frac{a_n+b_n}{2}]$ contient une infinité de termes de la suite, alors $[\frac{a_n+b_n}{2}, b_n]$ n'en contient aucun. On pose $a_{n+1}=a_n$ et $b_{n+1}=\frac{a_n+b_n}{2}$.
    \item Sinon, $[\frac{a_n+b_n}{2}, b_n]$ contient tous les termes de la suite à partir d'un certain rang. On pose $a_{n+1}=\frac{a_n+b_n}{2}$ et $b_{n+1}=b_n$.
\end{itemize}

On a $|a_n-b_n| = \frac{1}{2^n}$.

De plus, par construction, pour tout $n$ il existe un rang $N_n$ à partir duquel tous les termes de la suite sont dans $[a_n, b_n]$. Pour tout $m,p > N_n$ on a donc $|u_m-u_p|<\frac{1}{2^n}$.

Soit maintenant $\varepsilon > 0$. Soit $n$ le plus petit entier tel que $2^n>\frac{1}{\varepsilon}$. Il existe un rang $N$ à partir duquel $|u_m-u_p|<\frac{1}{2^n}$. Mais par définition, $\frac{1}{2^n}<\varepsilon$. D'où finalement, $(u_n)$ est de Cauchy.
\end{theoreme}

\begin{exo}
    Reprendre la démonstration ci dessus dans le cas décroissante et minorée, et ainsi achever la démonstration du théorème 1.
\end{exo}

\subsection{Une construction de $\mathbb{R}$ par les suites de Cauchy}
\subsubsection{Les réels}
On rappelle les notions suivantes :
\begin{defini}
Une relation d'équivalence sur E est une relation binaire $\sim$ sur E :
\begin{itemize}
\item réfléxive ($\forall x\in E, x\sim x$);
\item transitive ($(x\sim y \land y\sim z) \implies x\sim z$);
\item symétrique ($x\sim y \iff y\sim x$)
\end{itemize}

On appelle classe d'équivalence de $x$ l'ensemble noté $[x]=\{y\in E|y\sim x\}$. Remarquons que si $x\sim y$, $[x]=[y]$.
\end{defini}

\begin{prop}
    L'ensemble des classes d'équivalence est une partition de E.
\end{prop}

L'idée est de définir une relation d'équivalence $R$ sur les suites rationnelles :
$$(a_{n}) R (b_{n}) \iff a_{n}-b_{n} \to 0$$
On vérifie bien que c'est une relation d'équivalence :
\begin{itemize}
    \item $a_n-a_n=0 \to 0$,
    \item si $a_n-b_n \to 0$, alors $b_n-a_n = -(a_n-b_n) \to -0 =0$,
    \item si $a_n-b_n \to 0$ et $b_n-c_n \to 0$, alors $a_n-b_n+b_n-c_n = a_n-c_n \to 0$.
\end{itemize}

On peut donc partitionner $\mathcal{C}_{\mathbb{Q}}$ : cette partition est $\mathbb{R}$. Chaque classe d'équivalence est alors un réel, représenté par toutes les suites rationnelles qui l'approximent.

En identifiant tout rationnel $q$ à la classe d'équivalence de la suite stationnaire dont tous les termes sont égaux à $q$, $\mathbb{Q}\subset \mathbb{R}$.

\subsubsection{Leur structure algébrique}
On peut ensuite définir les opérations usuelles sur $\mathbb{R}$ :
\begin{defini}
    \begin{enumerate}
        \item $[(a_n)]+[(b_n)]=[(a_n+b_n)]$
        \item $[(a_{n})]\times [(b_{n})]=[(a_{n}b_{n})]$
    \end{enumerate}
 
    \tcblower
    Il faut ici vérifier que quelque soit la suite rationelle qu'on a choisi pour représenter un réel, l'addition et la multiplication donnera le même résultat. Autrement dit :
    \begin{enumerate}
        \item Si $(a_n)R(a'_n)$, $[(a_n+b_n)]=[(a'_n+b_n)]$.
        \item Si $(a_n)R(a'_n)$, $[(a_nb_n)]=[(a'_nb_n)]$.
    \end{enumerate}

    En effet comme attendu :
    \begin{enumerate}
    \item $a_n-a'_n = (a_{n}+b_{n})- (a'_{n}+b_{n})$, donc si $a_n-a'_n \to 0$, $(a_n+b_n)R(a'_n+b_n)$ soit $[(a_n+b_n)]=[(a'_n+b_n)]$.
    \item Si $a_n-a'_n \to 0$, comme $(b_n)$ est de Cauchy donc bornée, $b_n(a_n-a'_n) \to 0$.  D'où $(a_nb_n)R(a'_nb_n)$ soit $[(a_nb_n)]=[(a'_nb_n)]$.
    \end{enumerate}
\end{defini}

On définit de plus une relation d'ordre sur $\mathbb{R}$ :
\begin{defini}
Soit $x = [(a_n)] \in \mathbb{R}$. $x$ est positif si $x\neq 0$ et si il existe un rang $N$ tel que $\forall n \geq N, a_n > 0$. 
\tcblower
Il faut encore vérifier que cette définition a un sens, c'est à dire que si $a_n-b_n \to 0$ et $(a_n)$ ne tend pas vers 0, si $(a_n)$ finit par n'avoir que des termes positifs, alors $(b_n)$ aussi. 

Supposons que $\forall N, \exists n \geq N, b_n \leq 0$. $a_n-b_n \to 0$ c'est à dire $\forall \varepsilon >0, \exists N, n \geq N \implies  |a_n-b_n|<\varepsilon$.

Soit $\varepsilon >0$. $\exists N, n \geq N \implies  |a_n-b_n|<\frac{\varepsilon}{2}$. De plus, comme $(a_n)$ est de Cauchy, $\exists N', n, p \geq N' \implies |a_{n}-a_{p}| < \frac{\varepsilon}{2}$.

Enfin, $\exists m \geq \max(N, N'), b_m \leq 0$.  Mais alors comme $b_{m}- \frac{\varepsilon}{2}<a_m<b_{m}+ \frac{\varepsilon}{2}$, $\forall n \geq \max(N, N'), |a_n-a_m|< \frac{\varepsilon}{2}$, $b_m-\varepsilon<a_n<b_m+\varepsilon$ d'où $0<a_n\leq \varepsilon$. D'où $a_n \to 0$. 
\end{defini}

On achève maintenant la définition de la relation d'ordre :
\begin{defini}
    Soit $(x,y)\in \mathbb{R}^2$. On dit que $x\geq y$ si $x-y$ est positif ou si $x=y$.
    \tcblower
    C'est bien une relation d'ordre :
    \begin{itemize}
        \item $x=x$ donc $x\geq x$,
        \item antisymétrie
        \item transitivité 
    \end{itemize}
\end{defini}

On peut maintenant montrer que $\mathbb{R}$ est un corps ordonné.

\subsection{Une autre construction de $\mathbb{R}$}
\subsubsection{Les réels, version 2}
\subsubsection{Leur structure algébrique}
\subsubsection{Ces deux constructions se valent : isomorphisme}
En fait, même si les objets sous-jacents ne sont pas les mêmes - d'une part, des ensembles de suites, d'autre part, des parties de $\mathbb{Q}$ - ce qui importe ici, c'est la structure.

\subsection{Les propriétés fondamentales de $\mathbb{R}$}
\subsubsection{$\mathbb{R}$ n'est pas dénombrable}
\subsubsection{La propriété de la borne supérieure}
\subsubsection{La complétude}
*Généralisation aux espaces métriques*
\subsubsection{Caractérisations diverses et variées}
- *Théorème des segments emboités*
- *Théorème de la limite monotone*
- *Théorème de Bolzano-Weierstrass*
- *Équivalences entre ces résultats*
- *Unicité de $\mathbb{R}$*

En conclusion :
\begin{theoreme}[Existence et unicité de $\mathbb{R}$]
    Il existe un corps totalement ordonné vérifiant la propriété de la borne supérieure, unique à isomorphisme près.    
\end{theoreme}

