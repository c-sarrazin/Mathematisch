\titre{Un peu de théorie des ensembles}{Théorie des ensembles naïve et ZFC}
\subsection{Une théorie sur des bases frêles}
\subsubsection{Manipuler des ensenbles}
Qu'est-ce qu'un ensemble ? Tout, et rien à la fois. C'est un objet formel sur lequel on se donne uniquement une chose : la relation d'appartenance, $\in$.
Comment alors en pratique construire et utiliser des ensembles ? On peut penser qu'un ensemble est simplement défini par la donnée d'une propriété, éventuellement en langage naturel : $x\in A \iff$ Quelque chose est vrai de $x$. On verra plus tard que cette approche connaît de graves problèmes qui motiveront une axiomatisation plus rigoureuse de la théorie des ensembles. Mais pour l'instant, elle suffit à explorer quelques notions de base et à introduire du vocabulaire.
\begin{defini}
	Soient $A$ et $B$ deux ensembles. On appelle union de $A$ et $B$ l'ensemble $A \cup B=\{x|x\in A \lor x \in B\}$, et intersection de $A$ et $B$ l'ensemble $A\cap B=\{x|x\in A \land x\in B\}$. 
\end{defini}
\begin{defini}
	Soient $A$ et $B$ deux ensembles. On dit que $A$ est inclus dans $B$, ce qu'on note $A\subset B$, quand $\forall x \in A, x\in B$.
\end{defini}
\begin{defini}
	Soit $A\subset B$. On appelle complémentaire de $A$ dans $B$ l'ensemble $B\backslash A = \{x\in B | \lnot x\in A\}$
\end{defini}
Un poil plus compliqué : définir le produit cartésien $A\times B$. Pour ce faire, il faut trouver une façon purement ensembliste de définir un couple ordonné (sachant qu'un ensemble n'est justement pas ordonné). Une des solutions possibles:
\begin{defini}[Couples de Kuratowski]
	On note couple $(x,y)$ l'ensemble $\{x, \{x,y\}\}$.
\end{defini}
Cette définition permet de préserver l'ordre de la paire : $(x,y)\neq (y,x)$.
\begin{defini}
	Le produit cartésien de deux ensembles $A$ et $B$, noté $A\times B$, est l'ensemble des couples $(x,y)$ avec $x\in A$ et $y\in B$.
\end{defini}

On peut maintenant définir de façon purement ensembliste une relation binaire. 
\begin{defini}[Relation binaire]
	Soit E un ensemble. On appelle relation binaire $\mathcal{R}$ sur $(E, F)$ (ou si $E=F$, tout simplement sur $E$) un sous ensemble de $E\times F$. On dit que le couple $(x,y)$ de $E\times F$ vérifie la relation $\mathcal{R}$, ce qu'on note $x\mathcal{R}y$, quand $(x,y)$ appartient à cet ensemble $\mathcal{R}$.
\end{defini}
	
Intuitivement, une fonction de $E$ dans $F$ est un objet qui à tout $x$ de $E$ associe un unique $y$ de $F$. On formalise ceci à l'aide d'une relation binaire ayant une certaine propriété.

\begin{defini}
Soit $\phi$ une relation binaire sur $(E,F)$. $f$ est une application de $E$ dans $F$ quand $\forall(x,y,z)x\phi y \land x\phi z \implies y=z$. On notera à l'avenir $\phi(x)=y$. L'unicité de l'image est essentielle pour que cette notation est un sens : si  $\phi(x)=y$ et  $\phi(x)=z$, alors $y=z$.
\end{defini}

On définit encore diverses propriétés sur ces relations. 

	\begin{defini}[Premières définitions]
	Soit $\mathcal{R}$ une relation binaire de $E$. On dit que :
	\begin{itemize}
		\item $\mathcal{R}$ est réflexive si $\forall x\in E,$ $x\mathcal{R}x$.
		\item $\mathcal{R}$ est symétrique si $\forall (x,y)\in E^2,$ $x\mathcal{R}y\implies y\mathcal{R}x$.
		\item $\mathcal{R}$ est antisymétrique si $\forall (x,y)\in E^2,$ ($x\mathcal{R}y$ et $y\mathcal{R}x$) $\implies x=y$
		\item $\mathcal{R}$ est transitive si $\forall (x,y,z)\in E^3,$ ($x\mathcal{R}y$ et $y\mathcal{R}z$) $\implies x\mathcal{R}z$.
	\end{itemize}
\end{defini}
	\begin{defini}[Relation d'équivalence]
	On dit qu'une relation binaire $\mathcal{R}$ sur E est une relation d'équivalence si elle est réflexive, symétrique et transitive. On la note en général $\sim$.
    \end{defini}
	\begin{defini}[Relation d'ordre]
	On dit qu'une relation binaire $\mathcal{R}$ sur E est une relation d'ordre si elle est réflexive, antisymétrique et transitive. On note en général $\leq$ plutôt que $\mathcal{R}$.
    \end{defini}

\subsubsection{Familles et applications}

On revient maintenant aux applications.
\begin{defini}
	Soit $f: E \to F$ une application.
	\begin{itemize}
		\item $f$ est injective (ou est une injection) quand $\forall (x,y) \in E^2, f(x)=f(y) \implies x=y$.
		\item $f$ est surjective (ou est une surjection) quand $\forall x \in F, \exists y \in E, f(y)=x$.
		\item $f$ est bijective (ou est une bijection) quand elle est injective et surjective.
	\end{itemize}
\end{defini}

\begin{prop} \label{recolle}
	Soient $A, B, A', B'$ tels que $A\cap B = \varnothing$ et $A'\cap B' = \varnothing$. Soient $f : A \to A'$ et $g : B \to B'$ deux injections. La fonction $h : A \cup B \to A' \cup B'$ telle que $h_{/A}=f$ et $h_{/B}=g$ est injective.
	\tcblower
	Soient $x$ et $y$ tels que $h(x)=h(y)$. Supposons $h(x)=h(y)\in A'$. Alors $f(x)=f(y)$ d'où $x=y$. Un argument similaire tient pour $h(x)=h(y)\in B'$.
\end{prop}

\begin{defini}
	On dit que deux ensembles sont équipotents si il existe une bijection de l'un dans l'autre.
\end{defini}

Cette notion d'équipotence est la "bonne" notion pour définir le cardinal (la "taille") d'un ensemble, comme on le verra plus tard.  

\begin{theoreme}[Théorème de Cantor-Bernstein]
	Soient $A$ et $B$ deux ensembles. Si il existe une injection de $A$ dans $B$ et une injection de $B$ dans $A$, alors $A$ et $B$ sont équipotents.
	\tcblower
	Soient  $f$ et $g$ ces deux injections. On pose $B'= \{g(x)|x\in B\} \subset A$. $g$ est une bijection de $B$ vers $B'$. Construisons une bijection $h$ de $A$ vers $B'$. 

	On pose :
	$$\begin{cases}
		A_0 = A\backslash B' \\
		A_{n+1}=g\circ f(A_n)
	\end{cases}$$

	On définit :
	$$ h : A \to B$$
	$$a \mapsto \begin{cases}g \circ f(a) \text{ si } a \in \bigcup_{i\geq 0}A_i \\ a \text{ sinon}\end{cases}$$

	On a bien $\bigcup_{i\geq 0}A_i \cap A\backslash \bigcup_{i\geq 0}A_i=\varnothing$ et $\bigcup_{i\geq 1}A_i \cap A\backslash \bigcup_{i\geq 0}A_i = \varnothing$. De plus $g\circ f$ et la fonction identité sont injectives. D'où d'après \hyperref[recolle]{la propriété montrée plus haut}, $h$ est injective.

	Enfin $h(A)=B'$ donc $h$ est bien la bijection recherchée. Une bijection de $A$ vers $B$ est donc $g^{-1} \circ h$, et la preuve est achevée.
\end{theoreme}

\begin{theoreme}[Théorème de Cantor]
	Soit $E$ un ensemble. $E$ et $\mathcal{P}(E)$ ne sont pas équipotents.
	\tcblower
	Soit $f : E \to \mathcal{P}(E)$. Montrons que $f$ n'est pas surjective.

	Soit $F=\{x\in E|x\not \in f(x)\}$. Soit $x\in E$, deux cas se présentent :
	\begin{itemize}
		\item $x\in B$, alors $x\not\in f(x)$ et donc $x\in B\backslash f(x)$. D'où $B\backslash f(x)\neq\varnothing$ et donc $B\neq f(x)$.
		\item $\not \in B$, alors $x\in f(x)$ et donc $x\in f(x)\backslash B$. D'où $f(x)\backslash B\neq\varnothing$ et donc $B\neq f(x)$.
	\end{itemize}
	Finalement, $x$ ne peut être un antécédent pour $E$. Donc $E$ n'a pas d'antécédent et $f$ n'est pas surjective, donc pas bijective.

	On peut en bonus montrer qu'il n'y a pas d'injection de $\mathcal{P}(E)$ dans $E$. En effet comme l'application qui à $x$ associe $\{x\}$ est une injection de $E$ dans $\mathcal{P}(E)$ dans $E$, si il existait une injection de $\mathcal{P}(E)$ dans $E$, d'après le théorème de Cantor-Bernstein, $\mathcal{P}(E)$ et $E$ seraient équipotents.
\end{theoreme}

\subsubsection{Relation d'ordre et d'équivalence}

\subsection{Axiomatisation de Zermelo-Frankel (ZFC)}
\subsubsection{Motivation : un inventaire de paradoxes}
On a précédemment considéré qu'un ensemble pouvait être défini par la donnée de n'importe quelle propriété en langage naturel. Voyons pourquoi cette position est intenable.

\begin{boite}{Paradoxe de Berry}
	Soit $E$ l'ensemble des mots pouvant être définis en moins de 17 mots en français. On considère "le premier mot par ordre alphabétique ne pouvant pas être défini en moins de 17 mots français" : on vient de le définir en 17 mots, donc il peut bien être défini. La seule issue est de décréter que $E$ ne peut pas exister. 
\end{boite}

Ce paradoxe rend apparent la nécessité de s'en tenir au langage formel, tel que défini par \hyperref[lang]{le chapitre 1}. Mais ce n'est pas la seule limitation sur les ensembles pouvant exister.

On suppose maintenant qu'on peut définir un ensemble par la donnée d'une propriété en langage formel. Apparaît alors un autre paradoxe.

\begin{boite}{Paradoxe de Russel}
	Soit $A=\{x|x\not\in x\}$ l'ensemble des ensembles n'appartenant pas à eux mêmes. Est-ce que $A\in A$ ?

	Si $A\in A$, alors $A\not \in A$. Inversement, si $A \not\in A$, alors $A\in A$. Il en ressort comme préceddement qu'un tel ensemble n'existe nécessairement pas. 
\end{boite}

En fait, le problème vient du fait qu'on avait supposé la chose suivante
\begin{boite}{Axiome de compréhension}

\end{boite}

On va maintenant postuler la chose suivante
\begin{boite}[Axiome de séparation]
	
\end{boite}

Pour pouvoir éviter le paradoxe de Russel, il est aussi nécessaire que "l'ensemble de tous les ensembles" n'existe pas, aucun cas il suffirait de lui appliquer l'axiome de séparation pour retrouver l'axiome de compréhension.

\subsubsection{Axiomes de construction et de fondation}

Toute propriété ne définissant pas un ensemble, et encore moins en langage humain, on voit donc qu'il est nécessaire de se donner des axiomes plus rigoureux. En particulier, en se privant de l'axiome de compréhension, il est nécessaire d'introduire des axiomes pour pouvoir construire de nouveaux ensembles à partir d'anciens grâce aux opérations précedemment définies (réunion, intersection, produit cartésien...)

\begin{boite}{Axiomes de construction}
	\begin{itemize}
		\item Axiome d'extensionnalité : Deux ensembles sont égaux si ils ont exactement les même éléments, c'est à dir si appartenir à l'un est équivalent à appartienir à l'autre.
		\item Axiome de la paire : étant donné deux ensembles $A, B$, l'ensemble $\{A, B\}$ existe.
		\item Axiome de la réunion : étant donné deux ensembles $A, B$, l'ensemble $\{A, B\}$ existe.
		\item Axiome de l'ensemble des parties : étant donné un ensemble $A$, $\mathcal{P}(A)$ existe.
	\end{itemize}
\end{boite}

On y adjoint un axiome provisoire qui assure l'existence d'au moins un ensemble, par mesure de commodité. Cependant l'introduction dans le prochain chapitre d'un axiome d'existence vraiment utile (qui garantira l'existence de $\N$) nous conduira à l'abandonner. Notons que cette existence implique l'existence de l'ensemble vide par application de l'axiome de séparation avec une propriété toujours fausse (comme $x\neq x$).

On a maintenant construit toutes sortes d'ensembles : $\varnothing, \{\{\{\varnothing, \{\varnothing\}\}\}, \varnothing, \{\{\varnothing\}\}\}$... sur lesquels on peut vérifier nos précédentes définitions.

\begin{prop}
	Si $A$ est un ensemble, le singleton $\{A\}$ existe.
	\tcblower
	C'est l'ensemble "$\{A, A\}$" existant d'après l'axiome de la paire. 
\end{prop}

\begin{prop}
	L'intersection de deux ensembles est bien définie.
	\tcblower
	C'est l'ensemble $\{x\in A\cup B| x\in A \land x\in B\}$ existant d'après l'axiome de séparation.
\end{prop}

\begin{prop}
	Le produit cartésien de deux ensembles est bien défini.
	\tcblower
	Soit $A$ et $B$ deux ensembles. On a déjà dit que le couple $(x,y)$ était l'ensemble $\{\{x\}, \{x,y\}\}.$ Ce couple est donc un élement de $\mathcal{P}(\mathcal{P}(A\cup B))$. On définit ensuite le produit cartésien par séparation : $A\times B = \{z\in \mathcal{P}(\mathcal{P}(A\cup B))| \exists x \in A \exists y \in B z=\{\{x\}, \{x,y\}\}\}$.
\end{prop}

%to do : propriété caractéristique ????
\subsubsection{L'axiome du choix}

\subsubsection{Conclusion}
On vient de voir ce qu'on peut faire, comme opération basique, sur un ensemble - mais pour le moment il n'est pas clair comment on peut représenter des "vrais" objets mathématiques - des entiers, des réels... - avec. Dans le prochain chapitre, on développera la notion d'ordinal et de cardinal pour les ensembles infinis. On construira aussi les fondations de l'arithmétique. 