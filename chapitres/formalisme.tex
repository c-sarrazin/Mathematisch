\titre{Formalisme et raisonnement}{Un peu de théorie des ensembles}
\subsection{Relations binaires}
\begin{defini}[Relation binaire]
	Soit E un ensemble. On appelle relation binaire $\mathcal{R}$ sur E une condition d'appartenance à un certain sous ensemble de $E\times E$ appelé graphe de la relation $\mathcal{R}$. On dit que le couple $(x,y)$ de $E\times E$ vérifie la relation $\mathcal{R}$,noté $x\mathcal{R}y$ si et seulement si $(x,y)$ appartient au graphe de $\mathcal{R}$.
\end{defini}
	\begin{ex}
    Sur $\N$, $x\mathcal{R}y\iff x<y$ est une relation binaire
    \end{ex}
	
	\begin{defini}[Premières définitions]
	Soit $\mathcal{R}$ une relation binaire de $E$. On dit que :
	\begin{itemize}
		\item $\mathcal{R}$ est \textbf{réflexive} si $\forall x\in E,$ $x\mathcal{R}x$
		\item $\mathcal{R}$ est \textbf{symétrique} si $\forall (x,y)\in E^2,$ $x\mathcal{R}y\implies y\mathcal{R}x$
		\item $\mathcal{R}$ est \textbf{antisymétrique} si $\forall (x,y)\in E^2,$ ($x\mathcal{R}y$ et $y\mathcal{R}x$) $\implies x=y$
		\item $\mathcal{R}$ est \textbf{transitive} si $\forall (x,y,z)\in E^3,$ ($x\mathcal{R}y$ et $y\mathcal{R}z$) $\implies x\mathcal{R}z$
	\end{itemize}
\end{defini}
	\begin{defini}[Relation d'équivalence]
	On dit qu'une relation binaire $\mathcal{R}$ sur E est une relation d'équivalence si elle est \textbf{réflexive},\textbf{symétrique} et \textbf{transitive}\\
    \end{defini}
	\begin{defini}[Relation d'ordre]
	On dit qu'une relation binaire $\mathcal{R}$ sur E est une relation d'ordre si elle est \textbf{réflexive},\textbf{antisymétrique} et \textbf{transitive}, on note en général $\leq$ plutôt que $\mathcal{R}$
    \end{defini}