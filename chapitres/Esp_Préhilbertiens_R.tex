\subsection{Produit Scalaire}
\begin{defini}[Forme bilinéaire symétrique]
    Soit $E$ un espace vectoriel réel  et $\varphi : E\times E \to \R; (x,y)\mapsto \varphi(x,y)$. On dit que $\varphi$ est une forme bilinéaire symétrique si $\forall (x,y)\in E\times E, \varphi(x,y)=\varphi(y,x)$ et que $\forall y_0\in E,$ l'application $x \mapsto (\varphi(x,y_0))$ est linéaire.
\end{defini}
\begin{defini}[Positivité d'une fbs]
    Soit $\varphi$ une forme bilinéaire positive, elle est dite positive si $\forall x\in E, \varphi(x,x)\leq 0$.
\end{defini}
\begin{defini}[Définition positive d'une fbs]
    Soit $\varphi$ une forme bilinéaire positive, elle est dite defini positive si elle est positive et si  de plus $\forall x\in E, \varphi(x,x)=0\implies x=0$.
\end{defini}
\begin{defini}[Produit scalaire]
    On appelle produit scalaire de $E$ toute forme bilinéaire symétrique definie positive.
    Soit $\varphi$ un produit scalaire, on notera pour $x,y\in E$ plutôt $(x\mid y)$ ou $ \langle x,y\rangle $ que $\varphi(x,y)$
\end{defini}
\begin{defini}[Espace préhilbertien réel]
    On appelle espace préhilbertien réel tout espace vectoriel réel muni d'un produit scalaire
\end{defini}
\begin{defini}[Espace euclidien]
    On appelle espace euclidien tout espace préhilbertien réel de dimension finie
\end{defini}
\begin{prop}[Décomposition dans une base]
    Soit $e=(e_1,\dots,e_n)$ une base de $E$ espace euclidien, et $x,y\in E$ tel que $\displaystyle x=\sum_{k=1}^n x_k e_k$ et $\displaystyle y=\sum_{k=1}^n y_k e_k$ alors $\displaystyle(x\mid y)=\sum_{(i,j)\in \N_n^2}x_iy_j(e_i\mid e_j)$
    \tcblower
    Cette propriété est une application immédiate de la bilinéarité du produit scalaire.
\end{prop}
\begin{defini}[Norme euclienne]
    Soit $\varphi$ un produit scalaire sur $E$, on definit l'application norme euclidienne de $\varphi$, $E\to\R; x\mapsto \sqrt{(x\mid x)}$, la norme pour $x\in E$ se note $||x||$.
\end{defini}
\begin{defini}[Distance (euclidienne)]
    Soit $E$ préhilbertien réel, on definit l'application distance $$d: E \times E \to \R; (x,y)\mapsto ||x-y||=\sqrt{(x-y\mid x-y)}$$
\end{defini}
\begin{defini}[Vecteur normé]
    Un vecteur $x\in E$ est dit normé si $||x||=1$
\end{defini}
\begin{prop}[Scalaire d'un vecteur nul]
    Soit $E$ préhilbertien réel, alors $\forall(x,y)\in E^2, xy=0\implies (x\mid y)=0$
    \tcblower
    Comme le produit scalaire est symétrique on peut supposer $x=0$, alors $x=-x$ par exemple et par linéarité $(x\mid y)=-(x\mid y)=0$
\end{prop}
\begin{prop}[Sur la norme]
    \begin{itemize}
        \item $\forall x\in E, ||x||$
        \item $\forall x\in E, ||x||=0 \iff x=0$
        \item $ \forall x\in E, \forall \lambda \in \R, ||\lambda x||=|\lambda| ||x||$
    \end{itemize}
    \tcblower
    \begin{itemize}
        \item $||x||=\sqrt{(x\mid x)}$ or $(x\mid x)\geq 0$ par positivité du produit scalaire et on a bien $||x||\geq 0$
        \item $||x||=\sqrt{(x\mid x)}=0\iff (x\mid x)=0$ or le produit scalaire est defini positif donc $x=0$, réciproquement $||0||=0$.
        \item $||\lambda x||=\sqrt{(\lambda x\mid \lambda x)}$ puis par bilinéarité $||\lambda x||=\sqrt{\lambda^2( x\mid x)}=|\lambda| ||x||$.
    \end{itemize}
\end{prop}
\begin{prop}[Inégalité de Cauchy-Schwarz]
    Soit $E$ préhilbertien réel alors $$\forall (x,y)\in E^2, |(x\mid y)|\leq ||x|| ||y||$$
    De plus il y a égalité ssi $x$ et $y$ sont liés.
    \tcblower
    On pose ,pour $a,b\in E$, le polynôme $P$ sur $\R [X]$, $P(X)=||a+Xb||^2\geq 0$ en développant $(a+Xb\mid a+Xb)=||a||^2+2X(a\mid b)+X^2||b||$.
    Si $||b||=0,b=0$ et $(a\mid b)=0$, facilement on a $(a\mid b)\leq ||a|| ||b||$
    Sinon on associe au polynome $P$ sa fonction polynomiale sur $\R$ qui est une fonction du 2ème degrès, positive sur $\R$ donc de determinant négatif ou nul; $\Delta = 4X^2(a\mid b)^2-4X^2||a||^2||b||^2\leq 0$ et donc $(a\mid b)^2\leq ||a||^2||b||^2$ d'ou l'inégalité de Cauchy Schwarz.
    Pour le cas d'égalité, si $||b||=0,b=0$ et $(a,b)$ est liée (0 toujours liée), sinon on a forcement $\Delta=0$ et $P$ admet alors une racine double. $\exists x_0\in \R$ tel que $P(x_0)=||a+x_0b||^2=0$ alors $a+x_0b=0,a=-x_0b$ et $(a,b)$ est liée.
\end{prop}
\begin{prop}[Inégalité Triangulaire (ou de Minkowski)]
     
\end{prop}