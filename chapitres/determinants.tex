\titre{Groupe symétrique et déterminant}{Polynômes caractéristiques, théorème de Cayley-Hamilton}
\subsection{Le groupe symétrique}
\begin{defini}[Groupe symétrique (ou groupe des permutations)]
Pour tout entier naturel $n\geq 1$, on note l'ensemble fini $\N_n=\{1,\dots,n\}=\llbracket 1,n \rrbracket$.\\ On note alors $\mathfrak{S}_n$ ou $\mathcal{S}_n$ le groupe symétrique (ou groupe des permutations) d'indice $n$ qui correspond au groupe de toutes le permutations de $\N_n$, c'est à dire toutes les bijections de $\N_n$ sur lui-même.
\end{defini}

Une bijection $\sigma$ de $\mathfrak{S}_n$, c'est à dire une permutation, est une application de $\N_n$ dans $\N_n$ représente par
$$\sigma = \begin{pmatrix}
    1 & 2 & 3 & \cdots & n-1 & n \\
    \sigma(1) & \sigma(2) & \sigma(3) & \cdots &  \sigma(n-1)  & \sigma(n)
\end{pmatrix}$$
Avec $\N_n=\{\sigma(k)\mid k\in \N_n\}$.

\begin{ex}Dans $\mathfrak{S}_3$ une permutation possible serait :	
\[
\sigma = 
\begin{pmatrix}
    1 & 2 & 3 \\
    3 & 1 & 2
\end{pmatrix}
\]
Et alors $\sigma(1)=3$; $\sigma(2)=1$ et $\sigma(3)=2$\\
Les éléments de l'ensemble de départ étant toujours dans l'ordre on se permet parfois d'écrire uniquement $\sigma=	\begin{pmatrix}
    3 & 1 & 2
\end{pmatrix}$ ce qui signifie la même permutation.
\end{ex}


\begin{prop}[Structure de groupe]
    L'ensemble $\mathfrak{S}_n$ muni de la composition $\circ$ forme un groupe.
\tcblower
\begin{itemize}
\item Loi interne de composition: Soit $\sigma$ et $\sigma'$ deux permutations de $\mathfrak{S}_n$,
alors $\sigma'\circ\sigma$ et une bijection de $\N_n$ dans $\N_n$ par composé et $\sigma'\circ\sigma\in\mathfrak{S}_n$.

\item Élément neutre: Il est facile de voir que la fonction identité est un élément neutre, en effet $\forall k \in \N_n,\Id(k)=k$ et $\sigma\circ\Id(k)=\sigma(k)$ alors $\sigma\circ\Id=\sigma$
Réciproquement $\forall k \in \N_n,\Id(\sigma(k))=\sigma(k)$ et $\Id\circ\sigma=\sigma$.

\item Existence d'un inverse: Simplement pour toute permutation $\sigma$ on associe $\sigma^{-1}$ la permutation suivante : $\forall k \in \N_n$ $\sigma^{-1}(\sigma(k))=k$ c'est alors une bijection entièrement définie et $\sigma^{-1}\circ\sigma=\Id$. C'est alors la bijection réciproque de $\sigma$ et $\sigma\circ\sigma^{-1}=\Id$
\end{itemize}
\end{prop}

\begin{prop}[Cardinal de $\mathfrak{S}_n$]
L'ordre ou le cardinal de $\mathfrak{S}_n$, noté $\vert \mathfrak{S}_n \vert$ ou Card($\mathfrak{S}_n$) vaut $n!$
\tcblower
En effet une permutation $\sigma$ de $\mathfrak{S}_n$ est entièrement déterminée par le n-uplet $(\sigma(1),\dots,\sigma(n))$, on comprend facilement en commençant par 1, on a $n$ possibilités différentes pour $\sigma(1)$, puis alors $n-1$ pour $\sigma(2)$ car il ne peut plus prendre la valeur prise par $\sigma(1)$, puis $n-2$ possibilités pour $\sigma(3)$ et ainsi de suite.
On a donc bien $n!$ permutation distinctes dans $\mathfrak{S}_n$.
\end{prop}

\begin{ex}[$\mathfrak{S}_1$,$\mathfrak{S}_2$, et $\mathfrak{S}_3$]
\begin{itemize}
\item Pour $\mathfrak{S}_1$, comme $\N_1=\{1\}$, la seul permutation possible est $\sigma(1)=1$, $\sigma=\Id$ et $\mathfrak{S}_1=\{\Id\}$
\item Pour $\mathfrak{S}_2$ on a $2!=2$ permutation, trivialement $\sigma=\begin{pmatrix}
    1 & 2  \\
    1 & 2  
\end{pmatrix}=\Id$ et $\sigma'=\begin{pmatrix}
1 & 2  \\
2 & 1 
\end{pmatrix}$
\item Pour $\mathfrak{S}_3$ on a $3!=6$ permutations.

$\Id=\begin{pmatrix}
1 & 2 & 3  \\
1 & 2 & 3 
\end{pmatrix}$, $\sigma=\begin{pmatrix}
1 & 2 & 3  \\
3 & 1 & 2 
\end{pmatrix}$, $\sigma'=\begin{pmatrix}
1 & 2 & 3  \\
2 & 3 & 1 
\end{pmatrix}$, $\tau_1=\begin{pmatrix}
1 & 2 & 3  \\
2 & 1 & 3 
\end{pmatrix}$, $\tau_2=\begin{pmatrix}
1 & 2 & 3  \\
1 & 3 & 2 
\end{pmatrix}$, $\tau_3=\begin{pmatrix}
1 & 2 & 3  \\
3 & 2 & 1
\end{pmatrix}$
\end{itemize}
On peut alors se demander si le groupe est commutatif avec par exemple un table de groupe;
\end{ex}

\begin{ex}[Table de groupe (ou table de Pythagore ou table de Cayley)]
On va le faire avec $\mathfrak{S}_3$, pour le remplir on calcul les différents produits, on peut aussi utiliser le fait que tout les éléments doivent apparaitre exactement une fois dans chaque ligne ou colonne (comme un sudoku).\\
Pour les produits de permutation on applique l'une après l'autre.

$\tau_3\circ\sigma'=\begin{pmatrix}
3 & 2 & 1
\end{pmatrix}\begin{pmatrix}
2 & 3 & 1 
\end{pmatrix}=\begin{pmatrix}
2 & 3 & 1\\
3 & 2 & 1 
\end{pmatrix}=\begin{pmatrix}
1 & 3 & 2 
\end{pmatrix}=\tau_2$, (il faut s'entrainer)

Il faut faire suffisamment de calcul pour remplir le reste par sudoku, on obtient la table : 
\begin{center}
\noindent\begin{tabular}{c | c c c c c c}
$\circ$ & $\Id$ & $\tau_1$ & $\tau_2$ & $\tau_3$ & $\sigma$ & $\sigma'$  \\
\cline{1-7}
$\Id$ & $\Id$ & $\tau_1$ & $\tau_2$ & $\tau_3$ & $\sigma$ & $\sigma'$ \\
$\tau_1$ &$\tau_1$ & $\Id$ & $\sigma$ & $\sigma'$ & $\tau_2$ & $\tau_3$ \\
$\tau_2$ & $\tau_2$ & $\sigma'$ & $\Id$ & $\sigma$ & $\tau_3$ & $\tau_1$ \\
$\tau_3$ & $\tau_3$ & $\sigma$ & $\sigma'$ & $\Id$ & $\tau_1$ & $\tau_2$ \\
$\sigma$ & $\sigma$ & $\tau_3$ & $\tau_1$ & $\tau_2$ & $\sigma'$ & $\Id$ \\
$\sigma'$ & $\sigma'$ & $\tau_2$ & $\tau_3$ & $\tau_1$ & $\Id$ & $\sigma$\\
\end{tabular}
\end{center}
Ce qui se lit l'élément de la ligne fois celui de la colonne. On voit que la table n'est pas symétrique par rapport à la diagonale donc le groupe n'est commutatif.
\end{ex}

\begin{defini}[Orbite par permutation]
    On appelle orbite de $p\in\N_{n}$ d'une permuation $\sigma\in\mathfrak{S}_n$ l'ensemble $\{\sigma^k(p)\mid k\in\N\}$
\end{defini}

\subsection{Cycles et transpositions}
\begin{defini}[Cycle]
Dans $\mathfrak{S}_n$ avec $n\geq 2$, pour $p\geq 2$, $p\in\N_n$ on dit que $\sigma\in\mathfrak{S}_n$ est un cycle de longueur $p$ s'il existe $p$ éléments $a_1,a_2,\dots,a_p$ distincts de $\N_n$ tel que :
$\sigma(a_1)=a_2$, $\sigma(a_2)=a_3$, $\dots$ , $\sigma(a_{p-1})=a_p$ et $\sigma(a_p)=a_1$
Et que pour tout élément $b$ de $\N_n\backslash\{a_1,\dots,a_p\}$, $\sigma(b)=b$, on dit que $b$ est invariant par $\sigma$.
L'ensemble $\{a_1,a_2,\dots,a_p\}$ est appelé support du cycle $\sigma$, généralement on écrit ce cycle $(a_1,a_2,\dots,a_p)$.

Dans $\mathfrak{S}_n$, on appelle permutation circulaire un cycle de longueur $n$; càd de support $\N_n$.

Dans $\mathfrak{S}_1$, la seule permutation est l'identité, on peut la considéré comme un cycle à un seul élément.
\end{defini}

\begin{ex}
$\sigma=\begin{pmatrix}
1 &2 &3 &4 &5 &6 &7\\
5 &6 & 3 & 1 & 2& 4 & 7\\
\end{pmatrix}$ est le cycle $\begin{pmatrix}
1&5&2&6&4
\end{pmatrix}$
\end{ex}
\begin{defini}
    On dit que deux cycles $\sigma=(a_1,a_2,\dots,a_p)$ et $\sigma'=(b_1,b_2,\dots,b_q)\in\mathfrak{S}_n$ ont des supports disjoints si
    $\{a_1,a_2,\dots,a_p\}\cap\{b_1,b_2,\dots,b_q\}=0$
\end{defini}
\begin{prop}[Sur les cycles]
\begin{itemize}
\item l'inverse du cycle $\begin{pmatrix}
    a_1&a_2&\dots &a_p
\end{pmatrix}$ vaut $\begin{pmatrix}
a_p&a_{p-1}&\dots &a_1
\end{pmatrix}$
\item Soit $\sigma$ un cycle de longueur $p$ alors $\sigma^p=\Id$, on a fait un tour du cycle.
On en déduit que pour un entier relatif $m=pq+r$, $\sigma^m=\sigma^r$.
\item Deux cycle à support disjoint commutent
\end{itemize}
\end{prop}
\begin{prop}[Orbite d'un cycle]
    Un cycle est entièrment défini par l'orbite d'un élément de son support.
    \tcblower
    Soit $p\in\N_n$ un element du support de $\sigma$ un cycle de longueur $n$ alors $\sigma=\begin{pmatrix}
        p& \sigma(p)& \sigma^2(p)&\dots &\sigma^{n-1}(p)
    \end{pmatrix}$
    Soit $p\in\N_n$ un élément dans le support de $\sigma$ et de $\sigma'$ tel que $\{\sigma^k(p)\mid k\in\N\}=\{\sigma'^k(p)\mid k\in\N\}$ alors $\sigma=\sigma'$
\end{prop}
\begin{defini}[Transposition]
Dans $\mathfrak{S}_n$ avec $n\geq 2$, on dit que la permutation $\sigma\in\mathfrak{S}_n$ est une transposition si c'est un cycle de longueur 2, c'est à dire qu'il existe $i,j\in\N_n$ distinct tel que $\sigma(i)=j$ et $\sigma(j)=i$ et que $\forall k\in\N_n\backslash\{i,j\}$, $\sigma(k)=k$.\\
On note souvent cette transposition $\begin{pmatrix}
i & j
\end{pmatrix}$ ou $\begin{pmatrix}
j & i
\end{pmatrix}$ ou encore $\tau_{i,j}$.
\end{defini}
\begin{prop}[Sur les transposition]
    \begin{itemize}
        \item Comme on peut le voir avec la notation $\tau_{i,j}=\tau_{j,i}$, on a aussi facilement $\tau^2_{i,j}=\Id$ et donc $\tau_{i,j}=\tau^{-1}_{i,j}$.
        \item Pour avoir une transposition dans $\mathfrak{S}_n$, on prend deux éléments distincts de $\N_n$ c'est à dire $\binom{n}{2}=\frac{n(n-1)}{2}$ transpositions possibles.
        \item Soit $\begin{pmatrix}
            j & i
            \end{pmatrix}$ et $\begin{pmatrix}
                a & b
                \end{pmatrix}$ deux transpositions, $\begin{pmatrix}
                    j & i
                    \end{pmatrix}\begin{pmatrix}
                        a & b
                        \end{pmatrix}=\begin{pmatrix}
                            a & b
                            \end{pmatrix}\begin{pmatrix}
                                j & i
                                \end{pmatrix}\iff (\{a,b\}=\{j,i\} \vee \{a,b\}\cap\{j,i\})=\emptyset$ 
    \end{itemize}
\end{prop}
\begin{exo}
    On appelle centre de $\mathfrak{S}_n$ l'ensemble des éléments de $\mathfrak{S}_n$ qui commutent avec tous les autres.
    Déterminer pour $n\geq 3$ le centre de $\mathfrak{S}_n$
\end{exo}
\begin{theoreme}[Décomposition de permutations en produit de cycle]
    Toute permutation de $\mathfrak{S}_n$(avec $n\geq1$) se décompose en un produit de cycles à supports deux à deux disjoints. Cette décomposition est unique à l'ordre des facteurs près.
    
    \tcblower
    Existence par récurrence : $\mathcal{H}_n$:Toute permutation de $\mathfrak{S}_n$(avec $n\geq1$) se décompose en un produit de cycles à supports deux à deux disjoints.
    
    Pour $n=1$, la seule permutation est l'identité, c'est le cycle à $1$ éléments

    Soit $n$ tel que $\forall k\in\N_n$, $\mathcal{H}_k$.Soit $\sigma\in\mathfrak{S}_{n+1}$, on pose $A=\{\sigma^k(1)\mid k\in\N\}$ le support de $1$ par $\sigma$ et $B=\N_{n+1}\backslash A$.

    $A$ est un sous-ensemble de $\N_{n+1}$, il est donc de cardinal fini. Donc il existe une infinité d'entier $k\in\N$ tel que $\exists k'\in\llbracket 1,k-1\rrbracket,\sigma^{k'}(1)=\sigma^k(1)$.
    Comme tout sous-ensemble de $\N$ admet un plus petit élément, on peu posé $p$ le plus petit entier vérifiant $\exists p'\in\llbracket 1,p-1\rrbracket,\sigma^{p'}(1)=\sigma^p(1)$. On montre alors que $\sigma^p(1)=1$, en effet si $p'\geq 1,\sigma\circ\sigma^{p'-1}(1)=\sigma\circ\sigma^{p-1}(1)$ et par injectivité $\sigma^{p'-1}(1)=\sigma^{p-1}(1)$ ce qui contredit le caractère minimal de $p$.
    On se retrouve finalement avec $A=\{\sigma^k(1)\mid k\in\N_{p-1}\}$, on pose $c=(1,\sigma(1),\sigma^2(1),\dots,\sigma^{p-1}(1))$, et donc $\forall x\in A, c(x)=\sigma(x)$.
    Alors si $A=\N_{n+1}$, c'est à dire Card$(A)=n+1$, $c$ est un cycle donc on a $\mathcal{H}_{n+1}$
    Sinon on a $1\leq$Card$(B=\N_{n+1}\backslash A)\leq n$ car $1$ est forcement dans $A$, et $B$ invariant par $c$. On considérant $\sigma'$ la restriction de $\sigma$ à $B$, $\sigma'\in\mathfrak{S}_k$ avec $k\in\N_n$ car $\sigma'$ est injective comme restriction d'une application injective et $\sigma'(B)\subset B$, $k$ représente le cardinal de $B$ et par hypothèse de récurrences $\sigma'$ se décompose en produit de cycle à support disjoint notons les $c_2,\dots c_i$.
    Finalement $\sigma=c_1\circ c_2\circ\dots\circ c_i$ et $\mathcal{H}_{n+1}$.
    Unicité: Soit deux decompositions en cycles à supports disjoints de $\sigma$, $c_1\circ c_2\circ\dots\circ c_i$ et $c'_1\circ c'_2\circ\dots\circ c'_p$. Quitte à chambre la numérotation des cycles on peut supposer que 1 est dans le support de $c_1$ et de $c'_1$,
    alors $\forall k\in\N,c_1^k(1)=\sigma^k(1)=c'^k_1(1)$ et par propriété $c_1=c'_1$. On montre de manière analogue en prenant un élément commun au support de $c_2$ et $c'_2$ (quitte à changer la numérotation) et au final la décomposition est unique.
    
\end{theoreme}
\begin{theoreme}[Décomposition de cycle en produit de transposition]
    Tout cycle se décompose en produit de transposition, il en découle que toutes permutations (Sauf $\Id$ dans un esemble à 1 élément mais sans interets), se décompose en produit de transposition.
    \tcblower
    Soit $\begin{pmatrix}
        1 & 2 & \dots & n
        \end{pmatrix}$ un cycle, alors il se décompose par produit de transpositions $\begin{pmatrix}
            1 & 2
            \end{pmatrix}$$\begin{pmatrix}
                2 & 3
                \end{pmatrix}$
                $\begin{pmatrix}
                    3 & 4
                    \end{pmatrix}\dots$$\begin{pmatrix}
                        n-1 & n
                        \end{pmatrix}$
\end{theoreme}
\subsubsection{Signature d'une permutation}
\begin{defini}[Inversion]
    Soit $\sigma\in\mathfrak{S}_n$, et $(i,j)\in \N_n^2$, on dit que ce couple est une inversion si $i<j$ et $\sigma(i)>\sigma(j)$. On note $I(\sigma)$ le nombre d'inversion de $\sigma$.
\end{defini}
\begin{defini}[Signature d'un permutation]
    Soit $\sigma\in\mathfrak{S}_n$, en notant $I(\sigma)$ le nombre d'inversion de $\sigma$. On appelle signature(souvent noté $\varepsilon$) l'application

    $$\varepsilon : \mathfrak{S}_n\to\{-1,1\},\sigma\mapsto(-1)^{I(\sigma)}$$

    On peut le voir alors aussi en développant la définition de l'inversion que $\varepsilon(\sigma)=\prod_{1\leq i <j\leq n}sgn(\sigma(j)-\sigma(i))$
\end{defini}
\begin{theoreme}[Développement d'un déterminant]
    Blabla
\end{theoreme}
Pour une matrice $A\in\mathcal{M}_n(\K)$, $n\geq2$ de terme général $a_{ij}$ de déterminant $\Delta=\det A$, on donne les définitions suivantes :
\begin{defini}[Mineur d'un matrice $A$]
On appelle mineur $(i,j)$ de $A$ ou mineur de $a_{ij}$ dans $A$ le déterminant souvent noté $\Delta_{ij}$ ou $(\det A)_{ij}$ qui correspond au déterminant de $A$ ou l'on à supprimé la ligne $i$ et la colonne $j$.
\end{defini}
\begin{defini}[Cofacteur d'une matrice $A$]
On appelle cofacteur $(i,j)$ de $A$ ou  cofacteur de $a_{ij}$ dans $A$, souvent noté $\gamma_{ij}$ le scalaire  $\gamma_{ij}=(-1)^{i+j}\Delta_{ij}$ avec $\Delta_{ij}$ la mineur $(i,j)$ de $A$.
\end{defini}
\begin{defini}[Comatrice d'une matrice $A$]
La comatrice de $A$, noté $\text{Com}(A)$ est la matrice des cofacteurs de $A$, c'est a dire la matrice de terme générale $\gamma_{ij}$.
\end{defini}
\begin{prop}[Caractérisation suivant les lignes ou les colonnes d'un cofacteur]
(Rarement utilisé en pratique mais utile pour certaine démonstration),Soit $e=\{e_1,\dots,e_n\}$ une base canonique de $\K^n$, en désignant par $(C_1,C_2,\dots,C_n)$ le système des vecteurs colonnes de $A$ et 
par $(L_1,L_2,\dots,L_n)$ le système des vecteurs lignes de $A$ on a $\forall i,j\in\N_n$ le cofacteur:
$$\gamma_{ij}=\det_e(C_1,\dots,C_{j-1},e_i,C_{j+1},\dots,C_n)=\det_e(L_1,\dots,L_{j-1},e_j,L_{j+1},\dots,L_n)$$


\tcblower

On part de $D_C=\det_e(C_1,\dots,C_{j-1},e_i,C_{j+1},\dots,C_n)$ pour montrer l'égalité.
\[D_C=\begin{vmatrix}
a_{11} & a_{12} & \cdots &a_{1(j-1)}& 0 &a_{1(j+1)}& \cdots & a_{1n} \\
a_{21} & a_{22} & \cdots & a_{2(j-1)}& 0 &a_{2(j+1)}& \cdots & a_{2n} \\
\vdots & \vdots & \vdots & \vdots & \vdots & \vdots & \vdots & \vdots  \\
a_{(i-1)1} & a_{(i-1)2} & \cdots & a_{(i-1)(j-1)}& 0 &a_{(i-1)(j+1)}& \cdots & a_{(i-1)n} \\
0&0&\cdots&0&1&0&\cdots&0\\
a_{(i+1)1} & a_{(i+1)2} & \cdots & a_{(i+1)(j-1)}& 0 &a_{(i+1)(j+1)}& \cdots & a_{(i+1)n} \\
\vdots & \vdots & \vdots & \vdots & \vdots & \vdots & \vdots & \vdots  \\
a_{n1} & a_{n2} & \cdots &a_{n(j-1)}& 0 &a_{n(j+1)}& \cdots & a_{nn} \\
\end{vmatrix}\] On effectue un premier cycle sur les colonnes $\begin{pmatrix}
C_n & C_{n-1} & \cdots & C_{j+1} & C_j
\end{pmatrix}$ de longueur $n-j+1$ donc de signature $(-1)^{n-j}$.

Puis un deuxième sur les lignes $\begin{pmatrix}
L_n & L_{n-1} & \cdots & L_{i+1} & L_i
\end{pmatrix}$ de longueur $n-i+1$ donc de signature $(-1)^{n-i}$, la signature totale vaut $(-1)^{n-j+n-i}=(-1)^{-(j+i)}=(-1)^{j+i}$
On se retrouve alors avec $e_i$ en position $(n,n)$ et des zéros sur le reste de $C_n$ et $L_n$ alors 
{$$D_C=(-1)^{i+j}\begin{vsmallmatrix}
    a_{11} & a_{12} & \cdots &a_{1(j-1)}&a_{1(j+1)}& \cdots & a_{1n} & 0 \\
a_{21} & a_{22} & \cdots & a_{2(j-1)} &a_{2(j+1)}& \cdots & a_{2n}&0 \\
\vdots & \vdots & \vdots & \vdots & \vdots & \vdots & \vdots & \vdots  \\
a_{(i-1)1} & a_{(i-1)2} & \cdots & a_{(i-1)(j-1)} &a_{(i-1)(j+1)}& \cdots & a_{(i-1)n} & 0 \\
a_{(i+1)1} & a_{(i+1)2} & \cdots & a_{(i+1)(j-1)} &a_{(i+1)(j+1)}& \cdots & a_{(i+1)n} & 0\\
\vdots & \vdots & \vdots & \vdots & \vdots & \vdots & \vdots & \vdots  \\
a_{n1} & a_{n2} & \cdots &a_{n(j-1)} &a_{n(j+1)}& \cdots & a_{nn} & 0 \\
0&0&\cdots&0&0&\cdots&0&1
\end{vsmallmatrix}=(-1)^{i+j}D'_C$$}

Avec $D'_C$ le nouveau déterminant on calcul $\displaystyle D'_C=\sum_{\sigma\in\mathfrak{S}_n}\varepsilon(\sigma)\prod_{p=1}^n a_{\sigma(p),p}$

Or vu la $n$-ième ligne ou colonne $a_{\sigma(n),n}=0$ si $\sigma(n)\neq n$ et $a_{\sigma(n),n}=1$ si $\sigma(n)= n$.
\end{prop}

\begin{suite}
Or il existe un bijection de $\{\sigma\in\mathfrak{S}_n \text { tel que } \sigma(n)=n\}$ dans $\mathfrak{S}_{n-1}$ : $$ \varphi : \{\sigma\in\mathfrak{S}_n \mid \sigma(n)=n\}\to \mathfrak{S}_{n-1}$$
$$\sigma\mapsto \begin{pmatrix}
\N_{n-1}\to\N_{n-1}\\
i\mapsto \sigma(i)
\end{pmatrix}$$

Alors pour tout $\sigma\in \mathfrak{S}_n$ tel que $\sigma(n)=n$ il existe la permutation $\varphi(\sigma)\in\mathfrak{S}_{n-1}$ de décomposition en produit de cycles à support disjoint identique à $\sigma$, alors $\varepsilon(\sigma)=\varepsilon(\varphi(\sigma))$. De plus pour tout $\sigma\in\mathfrak{S}_{n-1}$ on peut associé la permutation $\varphi^{-1}(\sigma)\in\mathfrak{S}_n$ en posant pour $k\in\N_{n-1}$, $\varphi^{-1}(\sigma(k))=\sigma(k)$ et $\varphi^{-1}(\sigma(n))=n$, et $\varepsilon(\sigma)=\varepsilon(\varphi^-1(\sigma))$.

$\displaystyle D'_C=\sum_{\sigma\in\mathfrak{S}_{n-1}}\varepsilon(\sigma)\prod_{p=1}^{n-1} a_{\sigma(p),p}$
\end{suite}

\begin{theoreme}[Développement suivant une ligne ou une colonne]
Toujours pour une matrice $A\in\mathcal{M}_n(\K)$, $n\geq2$ de terme général $a_{ij}$ de déterminant $\Delta=\det A$, en utilisant les notations des définitions précédentes :

Pour tout $i\in\N_n$, on dit que l'on développe le déterminant suivant la $i$-ème ligne avec $\displaystyle \det A=\sum_{j=1}^n (-1)^{i+j}\Delta_{ij}a_{ij}=\sum_{j=1}^n \gamma_{ij}a_{ij}$.

Et pour tout $j\in\N_n$, on dit que l'on développe le déterminant suivant la $j$-ème colonne avec $\displaystyle \det A=\sum_{i=1}^n (-1)^{i+j}\Delta_{ij}a_{ij}=\sum_{j=1}^n \gamma_{ij}a_{ij}$.

\tcblower

Suivant une ligne $i$; soit $e=\{e_1,\dots,e_n\}$ une base canonique de $\K^n$.

$\displaystyle\det A =\det_e(L_1,\dots,L_{i-1},L_{i},L_{i+1},\dots,L_n)$ avec $\displaystyle L_i=\sum_{j=1}^n a_{ij}e_j$
donc $\displaystyle \det A =\det_e(L_1,\dots,L_{i-1},\sum_{j=1}^n a_{ij}e_j,L_{i+1},\dots,L_n)$ et par $n$-linéarité
$\displaystyle \det A =\sum_{j=1}^n a_{ij}\det_e(L_1,\dots,L_{i-1},e_j,L_{i+1},\dots,L_n)=\sum_{j=1}^n a_{ij}\gamma_{ij}$.
\end{theoreme}